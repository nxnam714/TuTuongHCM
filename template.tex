\documentclass{article}
\usepackage{vntex}

\usepackage{amsmath}
\usepackage{lastpage}
\usepackage[lined,boxed,commentsnumbered]{algorithm2e}
\usepackage{enumerate}
\usepackage{listings}
\usepackage{color}
\usepackage{graphicx}							% Standard graphics package
\usepackage{array}
\usepackage{tabularx, caption}
\usepackage{subcaption}
\usepackage{multirow}
\usepackage{multicol}
\usepackage{rotating}
\usepackage{graphics}
\usepackage{geometry}
\usepackage{setspace}
\usepackage{epsfig}
\usepackage{tikz}
\usepackage{indentfirst}
\usetikzlibrary{arrows,snakes,backgrounds}
\usepackage{hyperref}
\hypersetup{urlcolor=blue,linkcolor=black,citecolor=black,colorlinks=true} 



\title{NỘI DUNG ÔN TẬP THI CUỐI KỲ MÔN TƯ TƯỞNG HỒ CHÍ MINH\\ HỌC KỲ I (2018-2019) - CÁC LỚP HỆ CAO ĐẲNG VÀ ĐẠI HỌC} 

\begin{document}
	\maketitle
%%%%%%%%%%%%%%%%%%%%%%%%%%%%%%%%%%%%%%%%%%%%%%%%%%%%%%%%%%%%%%%%%%%%%%%%%%%%%%%%%%%%%%%%%%%%%%%%%%%%%%%%%%%%%%%%%%%%%%%%%%%%%%%%%%%%%%%%
	\section {Quá trình hình thành và phát triển của tư tưởng Hồ Chí Minh. Giá trị của tư tưởng Hồ Chí Minh đối với cách mạng Việt Nam và thế giới}
	Tư tường Hồ Chí Minh là một hệ thống quam điểm toàn diện và sâu sắc về những vấn đề cơ bản của cách mạng Việt Nam, kết quả của sự vận dụng và phát triển sáng tạo chủ nghĩa Mác-Lênin vào điều kiện cụ thể của nước ta, kế thừa và phát triển các giá trị truyền thống tốt đẹp của dân tộc, tiếp thu tinh hoa văn hóa nhân loại; là tài sản tinh thần vô cùng to lớn và quý giá của Đảng và dân tộc ta, mãi mãi soi đường cho sự nghiệp cách mạng của nhân dân ta giành thắng lợi.\\

	%%%%%%%%%%%%%%%%%%%%%%%%%%%%%%%%%%%%%%%%%%%%%%%%%%%%%%%%%%%%%%%%%%%%%%%%%%%%%%%%%%%%%%%%%%%%%%%%%%%%%%%%%%%%%%%%%%%%%%%%%%%%%%%%%%%%%%%%
	\subsection{Quá trình hinh thành tư tưởng Hồ Chí Minh có thể chia thành 5 giai đoạn tương ứng với mỗi bước phát triển.}
	\begin{enumerate}
		\item \textbf{Thời kì trước năm 1911: Hình thành tư tưởng yêu nước và chí hướng cứu nước}\\

		\textbf{Tác động từ gia đình:}\\
		Từ nhỏ, Nguyễn Sinh Cung đã có sự ảnh hưởng từ gia đình (cha, mẹ và hai anh chị) về lòng yêu nước, thương dân sâu sắc, về tinh thần lao động, ý chí kiên cường, về đức tính nhân hậu, đảm đang, sống chan hòa với mọi người.\\
		\textbf{Tác động từ xã hội:}\\
		Nghệ Tĩnh là vùng đất vừa giàu truyền thống văn hóa, vừa giàu truyền thống lao động, đấu tranh chống ngoại xâm.\\
		Từ nhỏ, Nguyễn Tất Thành chứng kiến cuộc sống nhèo khỏ, bị áp bức, bóc lột đến cùng cực của đồng bào mình. Tất cả thôi thúc Anh ra đi một con đường mới để cứu dân, cứu nước.\\
		\textbf{Hướng đi đúng của người:} Nguồn gốc những đau khổ và áp bức dân tộc là ngay ở "chính quốc", ở nước đế quốc đang thống trị mình.
		\item \textbf{Thời ký 1911-1920: Tìm thấy con đường cứu nước, giải phóng dân tộc}\\

		Đây là giai đoạn mang tính bước ngoặt trong tư tưởng Hồ Chí Minh. Trước đó, đã có rất nhiều người (như Phan Chu Trinh, Phan Bội Châu,...) đi tìm con đường giải phóng dân tộc, và từ 1911 người đã ra đi đi tìm đường cứu nước nhưng chưa ai tìm thấy con đường giải phóng thực sự. Kết thúc giai đoạn này, người đã tìm ra con đường đó.\\
		Ngày 2 5.6.1911, Người trên con tàu Latusơ Tơrêvin đi sang Pháp tìm đường cứu nước.
		Từ đó, người tiến hành các cuộc đã khảo sát mong tìm được chìa khóa cho chính dân tộc mình:\\
		\textbf{Khảo sát các cuộc cách mạng lớn thành công trên thế giới.}
		\begin{itemize}
			\item Với cách mạng tư sản (Pháp và Mỹ):\\
			Người đã đọc tuyên ngôn độc lập của Pháp và Mỹ sau cuộc cách mạng tư sản. Với Pháp đó là tuyên ngôn dân quyền và nhân quyền, đó là khẩu hiệu TỰ DO, BÌNH ĐẴNG, BÁC ÁI. Với Mỹ đó là quyền bình đẳng, quyền con người. Nhưng những điều đó lại mẫu thuẫn với đời sống của người dân tại Pháp, Mỹ. Với Mỹ, đó còn đó phân biệt chủng tộc, còn đó hành hình kiểu Linsơ. Với Pháp, người dân ở đó cũng bị bóc lột không khác gì tại nước ta. Người đã ví tư bản Pháp như con dỉa hai vòi. Từ đó, người kết luận: hai cuộc cách mạng của Pháp và Mỹ đã thành công nhưng thành công chưa tới.
			\item Với cách mạng vô sản (cách mạng Nga 1917):\\
			Đây là cuộc các mạng thành công của Nga. Kết quả của cách mạng này, người công dân (bị bóc lột ở xã hội tư sản) được chia sẻ tài sản.\\
			Theo Người, đây là cuộc cách mạng thành công tới nơi.
		\end{itemize}
		\textbf{Khảo sát bản chất của chủ nghĩa đế quốc: }\\
		Ngày 18.6.1919, Người gửi đến Hội nghị Vécxây \textit{Bản yêu sách của nhân dân An Nam} đòi quyền bình đẳng, tự do cho các dân tộc. Người gửi bản yêu sách tới Hội nghị không vì đòi những gì mình cần mà đòi những gì những nước tư bản đang có. Để thấy được bản chất từ chúng.
		Chúng chỉ là một lũ bịp bợm, nói nhưng không làm.\\
		Tới đây, người vẫn chưa tím thấy con đường giải phóng dân tộc cho đến ngày 16.7.1920, Người đọc Luận cương lần thứ nhất về vấn đề dân tộc và thuộc địa của của Lênin được đăng trên tờ báo Nhân đạo. Từ đây, Người nhận thấy: Chủ nghĩa chân chính nhất, chắc chắn nhất, cách mạng nhất chỉ có chủ nghĩa Mác-Lênin.\\
		Cùng phân tích một chút. Cách mạng của Nga thực chất là cuộc cách mạng vô sản, cách mạng giai cấp. Nó khác với cách mạng dân tộc của ta. Hoàn cảnh cùa Liên Xô và của ta cũng khác nhau. Nhưng Người vẫn chọn theo con đường chủ nghĩa Mác-Lênin bởi có một điểm chung rất lớn ở đây, đó chính là có chung kẻ thù, tất cả đều hướng tới đấu tranh chống lại các nước tư sản hay giai cấp tư sản.\\
		Sự kiện đánh dấu bước chuyển trong tư tưởng của Hồ Chí Minh đó là Người đã bỏ phiếu tán thành quốc thế III tại đại hội Tua. Đồng thời tham gia sáng lập Đảng cộng sản Pháp. Điều này thể hiện sự chuyển biến trong tư tưởng:
		\begin{itemize}
			\item Chuyển từ chủ nghĩa yêu nước sang chủ nghĩa Mác-Lênin.
			\item Chuyển từ giác ngộ dân tộc sang giác ngộ giai cấp.
			\item Chuyển từ một người yêu nước sang một chiến sĩ cách mạng, một người cộng sản.
		\end{itemize}
		Muốn cứu nước, giải phóng dân tộc không có con đường nào khác ngoài con đường cách mạng vô sản.
		\item \textbf{Thời kỳ 1921-1930: Hình thành cơ bản tư tưởng về cách mạng Việt Nam}\\

		Hồ Chí Minh là người đã đưa, việt hóa chủ nghĩa Mác-Lênin về Việt Nam. Vì hoàn cảnh Việt Nam và châu Âu khác nhau nên việc mang chủ nghĩa Mác-Lênin về Việt Nam cần có nội dung và hình thức phù hợp.\\
		\textbf{Về nội dung:} Người đã viết lại nội dung cho phù hợp với 6 nội dung cơ bản sau:
		\begin{itemize}
			\item Tính chất, nhiệm vụ, mục tiêu của cách mạng giải phóng dân tộc:\\
			Tính chất: Tính dân tộc.\\
			Nhiệm vụ: giải phóng dân tộc\\
			Mục tiêu: Đánh chủ nghĩa thực dân, giành độc lập dân tộc, xây dựng chính quyền nhân dân 5 giai cấp.
			\item Cách mạng giải phóng dân tộc muốn thắng lợi phải đi theo con đường cách mạng vô sản.
			\item Cách mạng giải phóng dân tộc trong thời đại mới phải do Đảng lãnh đạo, Đảng của giai cấp công nhân, nhân dân lao động và cả dân tộc Việt Nam.
			\item Lực lượng của cách mạng giải phóng dân tộc bao gồm toàn dân: Dù xã hội Việt Nam lúc bấy giờ có phân háo giai cấp song toàn bộ đều cùng chung một thân phận là "nô lệ mất nước" nên dễ dàng đoàn kết.
			\item Cách mạng giải phóng dân tộc cần được tiến hành chủ động, sáng tạo, và có khả năng giành thắng lợi trước cách mạng vô sản ở chính quốc.
			\item Cách mạng giải phóng dân tộc phải được tiến hành bằng con đưởng cách mạng bạo lực.
		\end{itemize}
		\textbf{Về hình thức:} Thời điểm đó, 90\% dân ta mù chữ nên hình thức truyền bá tư tưởng cần lựa chọn cho phù hợp.
		\begin{itemize}
			\item Viết sách, báo cho: Hướng tới đối tượng những nhà tri thức, người biết chữ để họ đọc, họ hiểu rồi họ truyền bá lại cho người khác.
			\item Vô sản hóa: Tức là Hội Việt Nam Cách mạng Thanh niên (75 người) rồi đưa họ trở về các nhà máy, xí nghiệp để họ truyền bá lại cho giai cấp công nhân.
		\end{itemize}
		\item \textbf{Thời kỳ 1930-1945: Vượt qua thử thách, kiên trì giữ vững lập trường cách mạng}\\

		Từ 1930-1937 là khoảng thời gian 8 năm mà Hồ Chí Minh bị đặt ra ngoài Đảng cộng sản Quốc tế vì: 6.1-7.2.1930, Người chủ trì hợp nhất ba tổ chức Đảng và đưa ra cương lình chính trị. Trong đó có 3 điểm mà cộng sản quốc tế không đồng tình:
		\begin{itemize}
			\item Đặt tên Đảng là Đảng cộng sản Việt Nam thay vì Đảng cộng sản Đông Dương.
			\item Đặt vấn đề dân tộc lên trên vấn đề giai cấp.
			\item Lực lượng giải cách mạng giải phóng là toàn dân thay vì giai cấp vô sản.
		\end{itemize}
		Từ đó dẫn đến hệ quả:
		\begin{itemize}
			\item Luận cương chính trị tháng 10 thay thế hoàn toàn cho Cương lĩnh chính trị.
			\item Đổi tên Dảng thành Đảng Cộng sản Đông Dương.
			\item Nguyễn Ái Quốc bị đặt bên ngoài Đảng 8 năm.
		\end{itemize}
		Cho đến 28.1.1941, Nguyễn Ái Quốc chủ trì Hội nghị Trung ương lần thứ tám và Ban chấp hành Trung ương Đảng Cộng sản Đông Dương đã hoàn chỉnh việc chuyển hướng chiến lược cách mạng Việt Nam.
		\item \textbf{Thời kỳ 1945-1969: Tư tưởng Hồ Chí Minh tiếp tục phát triển, hoàn thiện}
	\end{enumerate}

	%%%%%%%%%%%%%%%%%%%%%%%%%%%%%%%%%%%%%%%%%%%%%%%%%%%%%%%%%%%%%%%%%%%%%%%%%%%%%%%%%%%%%%%%%%%%%%%%%%%%%%%%%%%%%%%%%%%%%%%%%%%%%%%%%%%%%%%%
	\subsection{Giá trị của tư tưởng Hồ Chí Minh đối với cách mạng Việt Nam và thế giới}
	\subsubsection{Tư tưởng Hồ Chí Minh soi sáng con đường giải phóng và phát triển dân tộc}
	\begin{enumerate}
		\item Là tài sản tinh thần vô giá của dân tộc Việt Nam.
		\begin{itemize}
			\item Tư tưởng của Người không chỉ tiếp thu, kế thừa những giá trị, tinh hoa văn hóa, tư tưởng "vĩnh cửu" của loia2 người (chủ yếu là chủ nghĩa Mác-Lênin) mà còn đáp ứng nhiều vấn đề của thời đại, của sự nghiệp cách mạng Việt Nam và thế giới.
			\item Tư tưởng Hồ Chí Minh đã được kiểm nhiệm trong thực tiễn. Điều đó đảm bảo cho thắng lợi của cách mạng Việt Nam, đảm bảo cho tương lai, tiền đồ của dân tộc Việt Nam.
			\item Tư tưởng Hồ Chí Minh gồm những vấn đề xoay quynh việc giải phóng dân tộc và định hướng cho sự phát triển dân tộc.
		\end{itemize}
		\item Nền tảng tư tưởng và kim chỉ nam cho hành động của cách mạng Việt Nam.
		\begin{itemize}
			\item Tư tưởng Hồ Chí Minh đã trở thành ngọn cờ dẫn dắt cách mạng nước ta đi từ thắng lợi này đến thắng lợi khác.
			\item Tư tưởng Hồ Chí Minh giúp chúng ta nhận thức đúng những vấn đề lớn có liên quan đến việc bảo vệ nền độc lập dân tộc.
			\item Hồ Chí Minh đã đề ra lý luận về sự phát triển của các dân tộc giành độc lập tiến lên chủ nghĩa xã hội.
		\end{itemize}
	\end{enumerate}
	\subsubsection{Tư tưởng Hồ Chí Minh đối với sự phát triển thế gới}
	\begin{itemize}
		\item Tư tưởng Hồ Chí Minh phản ánh khát vọng thời đại.
		\item Tìm ra giải pháp đấu tranh giải phóng loài người.
		\item Cổ vũ các dân tộc đấu tranh trong sự nhiệp giải phóng.
	\end{itemize}

%%%%%%%%%%%%%%%%%%%%%%%%%%%%%%%%%%%%%%%%%%%%%%%%%%%%%%%%%%%%%%%%%%%%%%%%%%%%%%%%%%%%%%%%%%%%%%%%%%%%%%%%%%%%%%%%%%%%%%%%%%%%%%%%%%%%%%%%
	\section{Tư tưởng Hồ Chí Minh về vấn đề dân tộc}
	\subsection{Vấn đề dân tộc thuộc địa}
	\begin{enumerate}
		\item Thực chất của vấn đề dân tộc thuộc địa
		Vấn đề dân tộc là gì?
		\begin{itemize}
			\item Dân tộc ở đây được hiểu theo nghĩa là cộng đồng người có những điểm chung về văn hóa, đặc điểm sinh hoạt và có chung nhà nước.Hay cụ thể là dân tộc Việt Nam.
			\item Vấn đề là những điều không mong muốn nảy sinh và cần được giải quyết.
		\end{itemize}
		Như vậy, vấn đề dân tộc là mâu thuẫn, xung đột giữa các dân tộc hoặc giữa các quốc gia hay thực chất (theo quan điểm Hồ Chí Minh) là xung đột, mâu thuẫn giữa thực dân và nhân dân thuộc địa.\\

		Giải quyết vấn đề dân tộc thuộc địa:
		\begin{itemize}
			\item Đấu tranh chống chủ nghĩa thực dân, giải phóng dân tộc, giành độc lập dân tộc, thực hiện quyền dân tộc tự quyết, thành lập Nhà nước dân tộc độc lập.
			\item Lựa chọn con đường phát triển dân tộc: Hồ Chí Minh khẳng định phương hướng phát triển của dân tộc trong bối cảnh thời đại mới là chủ nghĩa xã hội.
		\end{itemize}

		\item Nội dung cốt lõi của vấn đề dân tộc thuộc địa: Độc lập dân tộc.
		\begin{itemize}
			\item Cách tiếp cận từ quyền con người\\
			Hồ Chí Minh hết sức trân trọng quyền con người. Từ quyền con người, Người đã khái quát và nâng cao thành quyền dân tộc: "Tất cả các dân tộc trên thế giới sinh ra đều bình đẳng; dân tộc nào cũng có quyền sống, quyền sung sướng và quyền tự do."
			\item Nội dung của độc lập dân tộc.\\
			Độc lập, tự do là khát vọng lớn nhất của các dân tộc thuộc địa.\\
			Độc lập dân tộc phải là một nền độc lập thật sự, hoàn toàn, gắn với hòa bình, thống nhất, toàn vẹn lãnh thổ của đất nước.\\
			Độc lập dân tộc, cuối cùng, phải đem lại cơm no, áo ấm, hạnh phúc cho mọi người dân.
		\end{itemize}
		\item Động lực: Chủ nghĩa yêu nước chân chính
		Động lực phải là chủ nghĩa yêu nước vì đó là đoàn kết toàn dân, tạo ra sức mạnh dân tộc chống lại kẻ thù. Song sức mạnh dân tộc chưa đủ. Đó phải là sức mạnh tổng hợp của sức mạnh dân tộc và sức mạnh thời đại (đoàn kết quốc tế). Bởi vậy phải là chủ nghĩa yêu nước chân chính, yêu dân tộc mình, vì độc lập dân tộc , vì nhiệm cụ cách mạng của dân tộc mình đồng thời tôn trọng, giúp đỡ các dân tộc sẵn sàng chung tay với mình, khác với chủ nghĩa vị quốc.
	\end{enumerate}
	\subsection{Mối quan hệ giữa vấn đề dân tộc và vấn đề giai cấp}
	\begin{enumerate}
		\item Vấn đề dân tộc và vấn đề giai cấp có mối quan hệ chặt chẽ với nhau.
		\begin{itemize}
			\item Hồ Chí Minh khẳng định vai trò của giai cấp công nhân và quyền lãnh đạo duy nhất của Đảng Cộng sản trong quá trình cách mạng Việt Nam.
			\item Thiết lập nhà nước của dân, do dân, vì dân.
			\item Gắn kết mục tiêu độc lập dân tộc với chủ nghĩa xã hội.
		\end{itemize}
		\item Giải phóng dân tộc là vấn đề trên hết, trước hết. Độc lập dân tộc gắn liền với chủ nghĩa xã hội.
		\begin{itemize}
			\item "Chỉ có chủ nghĩa xã hội, chủ nghĩa cộng sản mới giải phóng được các dân tộc bị áp bức và những người lao động trên thế giới khỏi ách nô lệ".
			\item "Nước độc lập mà dân không hưởng hạnh phúc tự do, thì độc lập cũng chẳng nghĩa lý gì". Do đó, sau khi giành độc lập, phải tiến lên xây dựng chủ nghĩa xã hội.
		\end{itemize}
		\item Giải phóng dân tộc là tiền đề để giải phóng giai cấp.\\
		Giải phóng dân tộc khỏi ách thông trị là tiền đề để giải phóng giai cấp. Vì thế, lợi ích của giai cấp phải phục tùng lợi ích của dân tộc.
		\item Giữ vững độc lập dân tộc mình, đồng thời tôn trọng độc lập của dân tộc khác.\\
		Nêu cao tinh thần độc lập, tự chủ, nhưng Hồ Chí Minh không quên nghĩa vụ quốc tế trong việc ủng hộ các cuộc đấu tranh giải phóng dân tộc trên thế giới.
	\end{enumerate}

	\section{Tư tưởng Hồ Chí Minh về xây dựng Đảng Cộng sản Việt Nam trong sạch, vững mạnh}
	\subsection{Xây dựng Đảng - Quy luật tồn tại và phát triển của Đảng}
	Xây dựng Đảng là quy luật tồn tại và phát triển, bởi vậy hoạt động xây dựng Đảng là hoạt động mà bất cứ Đảng nào cũng cần có.\\

	Hồ Chí Minh xem xây dựng Đảng là nhiệm vụ tất yếu, thường xuyên
	\begin{enumerate}
		\item Thường xuyên\\
		Xây dựng Đảng không phải là khi trong Đảng có gì đột biến hay trong Đảng "có vấn đề nổi cộm" mới cần đến một giải pháp tình thế. Xây dựng Đảng là nhiệm vụ vừa cấp bách, vừa lâu dài:
		\begin{itemize}
			\item Khi cách mạng gặp khó khăn: Xây dựng Đảng để củng cố lập trường, bản lĩnh, sáng suốt, không bị động, tránh bi quan.
			\item Khi cách mạng thuận lợi: Xây dựng tư tưởng, quan điểm cách mạng khoa học; ngăn ngừa chủ quan, tự mãn, tha hóa, biến chất.
		\end{itemize}
		\item Tất yếu\\
		Tính tất yếu khách quan của công tác xây dựng Đảng được Hồ Chí Minh lý giải theo căn cứ sau:
		\begin{itemize}
			\item Mỗi thời kỳ, giai đoạn lịch sử khác nhau có mục tiêu, nhiệm vụ, yêu cầu cách mạng riêng. Đảng phải tự đổi mới để thực hiện nhiệm vụ mới. Phải xây dựng Đảng để có trở nên nhạy bén, làm chủ thời cuộc.
			\item Đảng là một bộ phận của xã hội; cán bộ Đảng viên đều chịu ảnh hưởng, tác động của những cái tốt, cái xấu của xã hội. Do vậy, không rèn luyện, không xây dựng Đảng sẹ mất khả năng đề kháng trước cái xấu dẫn thới tha hóa.
			\item Xây dựng Đảng là cơ hội để mỗi đảng viên tự rèn luyện, giáo dục và tu dưỡng tốt hơn. Đây là cơ hội bởi theo Hồ Chí Minh, "vì điều kiện khó khăn, mà số đông cán bộ và đảng viên chưa được huấn luyện hẳn hoi".
		\end{itemize}
		"Một dân tộc, một đảng và mỗi con người, ngày hôm qua là vĩ đại, có sức hấp dẫn lớn, không nhất định ngày hôm nay và ngày mai vẫn được mọi người yêu mến và ca ngợi, nếu lòng dạ không trong sáng nữa, nếu sa vào chủ nghĩa cá nhân" - Hồ Chí Minh.
	\end{enumerate}
	\subsection{Nội dung công tác xây dựng Đảng Cộng sản Việt Nam}

	\begin{enumerate}
		\item Xây dựng Đảng về tư tưởng, lý luận\\
		Xây dựng Đảng về tư tường lý luận là xây dựng chủ nghĩa Mác-Lênin. Hồ Chí Minh khẳng định: "Đảng muốn vững thì phải có chủ nghĩa làm cốt, trong đảng ai cũng phải hiểu, ai cũng phải theo chủ nghĩa ấy. ... "chủ nghĩa" ấy là chủ nghĩa Mác-Lênin".
		\textbf{Cách xây dựng:}
		\begin{itemize}
			\item Việc học tập, nghiên cứu, tuyên truyền chủ nghĩa Mác-Lênin phải luôn phù hợp với từng đối tượng.
			\item Việc vận dụng chủ nghĩa Mác-Lênin phải luôn luôn phù hợp với từng hoàn cảnh. Phải tránh giáo điều, tránh xa rời với các nguyên tắc cơ bản của chủ nghĩa Mác-Lênin.
			\item Trong quá trình hoạt động, Đảng ta phải chú ý học tập, kế thừa những kinh nghiệm tốt của các đảng sản khác, đồng thời Đảng ta phải tổng kết kinh nghiệm của mình để bổ sung vào chủ nghĩa Mác-Lênin.
			\item Đảng ta phải tăng cường đấu tranh để bảo vệ sự trong sáng của chủ nghĩa Mác-Lênin. Chống giáo điều, cơ hội, xét lại chủ nghĩa Mác-Lênin.
		\end{itemize}
		\item Xây dựng Đảng vể chính trị.\\
		Xây dựng Đảng về chính trị có nhiều nội dung, trong đó, việc xây dựng đường lối là vấn đề cốt tử trong tồn tại và phát triển của Đảng.\\
		Đường lối chính trị phải dựa trên cơ sở lý luận của chủ nghĩa Mác-Lênin. Vận dụng vào hoàn cảnh cụ thể của nước ta, đường lối về kinh tế, chính trị, văn hóa - xã hội cần phù hợp và giải quyết được những vấn đề của cách mạng Việt Nam.
		\item Xây dựng Đảng về tổ chức, bộ máy, công tác cán bộ.
		\begin{itemize}
			\item Hệ thống tổ chức Đảng:\\
			Hồ Chí Minh khẳng định sức mạnh của Đảng bắt nguồn từ tổ chức.\\
			Trong hệ thống tổ chức Đảng, Hồ Chí Minh rất coi trọng vai trò của chi bộ. Chi bộ là tổ chức hạt nhân, quyết định chất lượng lãnh đạo của Đảng.\\
			Cần xây dựng chi bộ vững mạnh từ trung ương đến địa phương.
			\item Các nguyên tắc sinh hoạt Đảng.\\
			\begin{itemize}
				\item Tập chung dân chủ. Đây là nguyên tắc cơ bản trong xây dựng Đảng. Tập trung trên nền tảng dân chủ; dân chủ dưới sự chỉ đạo tập trung.
				\item Tập thể lãnh đạo, cá nhân phụ trách.\\
				Tập thể lãnh đạo: "Góp kinh nghiệm và sự xem xét của nhiều người, thì vấn đề đó mới được thấy rõ khắp mọi mặt. Mà có thấy rõ khắp mọi mặt, thì vấn đề ấy mới được giải quyết chu đáo, khỏi sai lầm".\\
				Cá nhân phụ trách: "Nếu không có cá nhân phụ trách thì sẽ sinh cái tệ người này ủy cho người kia, người kia ủy cho người nọ, kết quả là không ai thi hành. Như thế thì việc gì cũng không xong".
				\item Tự phê bình và phê bình\\
				Mục đích là để làm cho phần tốt trong mỗi con người nảy nở, tổ chức tốt lên, phần xấu bị mất dần đi, vươn tới chân, thiện, mỹ.\\
				Thái độ, phương pháp: phải tiến hành thường xuyên; phải thẳng thắn, chân thành, trung thực, không nể nang, không giấu giếm và cũng không thêm bớt khuyết điểm.
				\item Kỷ luật nghiêm minh, tự giác.
				\item Đoàn kết, thông nhất trong Đảng. Có đoàn kết tốt mới tạo ra cơ sở vững chắc để thống nhất ý chí và hành động.
			\end{itemize}
			\item Cán bộ, công tác cán bộ Đảng.\\
				Cán bộ là mắt khâu trung gian nối liền giữa Đảng, Nhà nước với nhân dân.
				Mọi việc thành hay bại là do cán bộ tốt hay kém.
				Công tác cán bộ bao gồm:
				\begin{itemize}
					\item tuyển chọn cán bộ;
					\item đào tạo, huấn luyện, bồi dưỡng cán bộ;
					\item đánh giá đúng cán bộ; tuyển dụng, sắp xếp, bố trí cán bộ;
					\item thực hiện các chính sách đối với cán bộ.
				\end{itemize}
		\end{itemize}
		\item Xây dựng Đảng về đạo đức
		\begin{itemize}
			\item Một Đảng chân chính cách mạng phải có đạo đức.
			\item Đạo đức của Đảng là đạo đức mới, đạo đức cách mạng, mang bản chất của giai cấp công nhân, cũng là đạo đức Mác-Lênin.
			\item Nội dung cốt lõi là chủ nghĩa nhân đạo chiến đấu.
		\end{itemize}

	\end{enumerate}
	\section{Quan điểm của Hồ Chí Minh về đại đoàn kết dân tộc}
	Đại đoàn kết dân tộc là tập hợp mọi lực lượng (ở bên trong) có thể tập hợp được để tạo nên sức mạnh dân tộc nhằm thực hiện thắng lợi nhiệm vụ cách mạng của dân tộc mình.\\

	\subsection{Vai trò của đại đoàn kết dân tộc}
	\begin{enumerate}
		\item Đại đoàn kết dân tộc là vấn đề có ý nghĩa chiến lược, quyết định thành công của cách mạng\\
		Ỹ nghĩa chiến lược tức là thể hiện tính nhất quán, lâu dài và xuyên suốt của đại đoàn kết dân tộc.\\
		Quyết định thành công: Nó mang tính sống còn, quyết định thành bại của cách mạng. Chẳng hạn: 
		\begin{itemize}
			\item Mặt trận Việt Minh (1941) quyết định thành công của cách mạng tháng 8. 
			\item Mặt trận Liên Việt(1846) quyết định thành công của công cuộc chống Pháp. 
			\item Mặt trận dân tộc giải phóng miền Nam Việt Nam(1960) quyết định thành công của công cuộc chống Mỹ.
		\end{itemize}
		Khẩu hiệu: Đoàn kết đoàn kết, đại đoàn kết. Thành công, thành công, đại thành công.
		\item Đại đoàn kết dân tộc là mục tiêu, nhiệm vụ hành đầu của Đảng, của dân tộc.\\
		Hàng đầu tức là phải thực hiện nhiệm vụ đoàn kết trước rồi mới thực hiện nhiệm vụ cách mạng (ở mỗi giai đoạn lịch sử khác nhau sẽ có những nhiệm vụ cách mạng khác nhau).\\
		Vai trò của Đảng: Đề ra đường lối đúng đắn, đồng thời phải cụ thể hóa thành những mục tiêu, nhiệm vụ và phương pháp cách mạng phù hợp.
	\end{enumerate}

	\subsection{Lực lượng đại đoàn kết dân tộc}
	\begin{enumerate}
		\item Đại đoàn kết dân tộc là đại đoàn kết toàn dân\\
		Toàn dân tức là bao gồm toàn bộ dân và nhân dân: "mọi con dân nước Việt", "mỗi một người con Rồng cháu Tiên", "Ai có tài, có đức, có sức, có lòng phụng sự Tổ quốc thì và phục vụ nhân dân thì ta đoàn kết với họ".
		\item Điều kiện thực hiện đại đoàn kết dân tộc
		\begin{itemize}
			\item Để xây dựng khối đại đoàn kết toàn dân phải kế thừa truyền thống yêu nước - nhân nghĩa - đoàn kết cuả dân tộc: Để đoàn kết thì cân tìm được điểm chung của toàn dân tộc. Lòng yêu nước, nhân nghĩa chính là điểm chung đó khi mà nước ta thời điểm đó đang đấu tranh chông tại Pháp và sau này là Mỹ.
			\item Phải có long khoan dung, độ lượng với con người. Cái đoàn kết mà Hồ Chí Minh hướng tới là đoàn kết thực sự, nó khác với việc bằng mặt không bằng lòng hay khác với thủ đoạn chính trị. Cần có lòng độ lượng bởi nhân vô thập toàn, ai cũng có ưu điểm và khuyết điểm. Cho nên, vì lợi ích cách mạng, cần trân trọng cái phần thiện dù lớn hay nhỏ trong mỗi con người.
			\item Để thực hành đoàn kết rộng rãi cần có niềm tin vào nhân dân. Cần tin tưởng vào dân bởi dân là lực lượng duy nhất có thể thực hiện thắng lợi nhiệm vụ cách mạng. "Khó vạn lần không dân cũng chịu. Khó vạn lần, dân liệu cũng xong".
		\end{itemize}
	\end{enumerate}

	\subsection{Hình thức tổ chức khối đại đoàn kết dân tộc}
	\begin{enumerate}
		\item Hình thức tổ chức khối đại đoàn kết dân tộc là mặt trận dân tộc thống nhất.\\
		Mặt trận dân tộc thống nhất là nơi quy tụ mọi tổ chức và cá nhân yêu nước, nơi tập hợp mọi người con dân nước Việt, không chỉ ở trong nước mà còn bao gồm cả những người Việt Nam định cư ở nước ngoài, dù ở bất cứ phương trời nào nếu tấm lòng vẫn hướng về quê hương, đất nước, về Tổ quốc Việt Nam, đều được coi là thành viên của mặt trận.
		\item Nguyên tắc cơ bản về xây dựng hoạt động của mặt trận dân tộc thống nhất.
		\begin{itemize}
			\item Mặt trận dân tộc thống nhất phải được xây dựng trên nền tảng khối liên minh công - nông - trí thức, đặt dưới sự lãnh đạo của Đảng.
			\item Mặt trận dân tộc thống nhất phải hoạt động trên cơ sở đảm bảo lợi ích tối cao của dân tộc, quyền lợi cơ bản của các tầng lớp nhân dân.
			\item Mặt trận dân tộc thống nhất phải hoạt động thống nhất phải hoạt động thep nguyên tắc hiệp thương dân chủ, đảm bảo đoàn kết ngày càng rộng dãi và bền vưỡng.
			\item Mặt trận dân tộc thống nhất là khối đoàn kết chặt chẽ, lâu dài, đoàn kết thật sự, chân thành, thân ái giúp đỡ nhau cùng tiến bộ.
		\end{itemize}
	\end{enumerate}

	\section{Quan điểm Hồ Chí Minh về văn hóa}
	\subsection{Quan điểm cơ bản của Hồ Chí Minh về văn hóa}
	\begin{enumerate}
		\item Định nghĩa về văn hóa và quan điểm về xây dựng nên văn hóa.
		\begin{enumerate}
			\item Định nghĩa về văn hóa\\
			"Vì lẽ sinh tồn cũng như mục đích của cuộc sống, loài người mới sáng tạo và phát minh ra ngôn ngữ, chữ viết, đạo đức, pháp luật, khoa học, tôn giáo, văn học, nghệ thuật, những công cụ cho sinh hoạt hằng ngày về mặc, ăn, ở và các phương thức sử dụng. Toàn bộ những sáng tạo và phát minh đó tức là văn hóa. Văn hóa là tổng hợp của mọi phương thức sinh hoạt cùng với biểu hiện của nó mà loài người đã sản sinh ra nhắm thích ứng những nhu cầu đời sông và đòi hỏi của sự sinh tồn".
			\item Xây dựng nền văn hóa mới bằng định nghĩa xã hội trên tất cả 5 lĩnh lực đời sống xã hội:
			\begin{itemize}
				\item Xây dựng tâm lý: tinh thần độc lập tự cường.
				\item Xây dựng luân lý: biết hi sinh mình, làm lợi cho quần chúng.
				\item Xây dựng xã hội: mọi sự nghiệp có liên quan đến phúc lợi của nhân dân trong xã hội.
				\item Xây dựng chính trị: dân quyền.
				\item Xây dựng kinh tế.
			\end{itemize}
		\end{enumerate}
		\item Quan điểm của Hồ Chí Minh về các vấn đề chung của văn hóa.
		\begin{enumerate}
			\item Vị trí, vai trò của văn hóa trong đời sống xã hội.
			\begin{itemize}
				\item Văn hóa là đời sống tin thần của xã hội, thuộc kiến trúc thượng tầng.\\
				Văn hóa chịu tác động, bị quy định bởi các lĩnh vực khác như chính trị, kinh tế, xã hội (thuộc cơ sở hạ tầng). Nói cách khác, mối quan hệ giữa văn hóa với chính trị, kinh tế, xã hội là mối quan hệ giữa cơ sở hạ tầng và kiến trúc thượng tầng.
				\item Văn hóa không thể dứng vững ngaoi2 mà phải ở trong kinh tế và chính trị, phải phục vụ nhiệm vụ chính trị và thúc đẩy sự phát triển của kinh tế. Như vậy, văn hóa vừa là nền tảng tinh thần, vừa là động lực phát triển của xã hội:
				\begin{itemize}
					\item Các lĩnh vực chính trị, kinh tế, xã hội tác động đến văn hóa.
					\item Văn hóa là động lực để phát triển kinh tế xã hội
				\end{itemize}
			\end{itemize}
			\item Tính chất của nền văn hóa 
			\begin{itemize}
				\item Tính dân tộc: văn hóa mang đặc trưng, bản sắc của môi dân tộc. Nó phân biệt, không nhầm lẫn với văn hóa của các dân tộc khác.
				\item Tính khoa học: tính hiện đại, tiên tiến, thuận với trào lưu tiến hóa của thời đại.
				\item Tính đại chúng: được thể hiện ở chỗ nền văn hóa phải phục vụ nhân dân và do nhân dân xây dựng nên.
			\end{itemize}
			\item Chức năng của văn hóa.
			\begin{itemize}
				\item Bồi dưỡng tư tưởng đúng đắn và những tình cảm cao đẹp.\\
				Tư tưởng có thể đúng đắn hoặc sai lầm, tình cảm có thể thấp hèn hoặc cao đẹp. Chức năng cao quý nhất của văn hóa là bồi dưỡng tư tưởng, lý tưởng đúng đắn (độc lập dân tộc gắn liền với chủ nghĩa xã hội) và tình cảm cao đẹp (lòng yêu nước, người bảo vệ cái đúng).
				\item Mở rộng hiểu biết, nâng cao dân trí\\
				Nâng cao hiểu biết của nhân dân trong tất cả các lĩnh vực đời sống xã hội để nhân dân tham gia sáng tạo và hưởng thụ văn hóa.
				\item Bồi dưỡng những phẩm chất, phong cách và lối sống tốt đẹp, lành mạnh; hướng con người đến chân, thiện, mỹ để hoàn thiện bản thân. Phong cách, phẩm chất của mỗi người được cấu thành từ văn hóa và nó được biểu hiện ở đời sống hàng ngày.
			\end{itemize}
		\end{enumerate}
	\end{enumerate}
	\subsection{Quan điểm của Hồ Chí Minh về một số lĩnh vực của văn hóa}
	\begin{enumerate}
		\item {Văn hóa giáo dục}\\
		Hồ Chí Minh đã xem xét và thấy được những sai lầm, thiếu xót,... của hai nền giáo dục tồn tại lúc bấy giờ là:
		\begin{itemize}
			\item Giáo dục phong kiến: tầm chương, kinh viện, xa rời thực tế, bất bình đẳng, trọng nam kinh nữ,...
			\item Giao dục thực dân: ngu dân, đồi bại, xảo trá, nguy hiểm hơn cả sự dốt nát.
		\end{itemize}
		Từ đó Hồ Chí Minh đã có quan điểm toàn diện về giáo dục.\\

		\textbf{Quan điểm Hồ Chí Mionh về văn hóa giáo dục}
		\begin{itemize}
			\item Mục tiêu của giáo dục: là để thực hiện ba chức năng của văn hóa thông qua việc dạy và học.
			\begin{itemize}
				\item Văn hóa giáo dục phải cải tạo tri thức cũ, đào tạo tri thức mới: Công nông hóa trí thức, trí thức hóa công nông.
				\item Văn hóa giáo dục phải đào tạo được những lớp người có đức, có tài kế tục sự nghiệp cách mạng.
			\end{itemize}
			\item Nội dung giáo dục: phải phù hợp với thực tiễn Việt Nam.
			\begin{itemize}
				\item Giáo dục toàn diện: văn hóa, chính trị, khoa học - kỹ thuật, chuyên môn nghiệp vụ.
				\item Tiến hành cải cách giáo dục thường xuyên cả về nội dung và hình thức để trở nên hợp lí và đáp ứng đòi hỏi của cách mạng.
			\end{itemize}
			\item Phương châm, phương pháp giáo dục:
			\begin{itemize}
				\item Phương châm:
				\begin{itemize}
					\item Học đi đôi với hành, lý luận phải đi đôi với thực tế.
					\item Phải kết hợp thật chặt chẽ ba khâu: gia đình, nhà trường và xã hội.
					\item Học mọi lúc, mọi nơi; học mọi người, học suốt đời.
					\item Coi trọng việc tự học, tự đào tạo và đào tạo lại.
				\end{itemize}
				\item Phương pháp: giáo dục phải phù hợp với mục tiêu giáo dục. Phải phù hợp với trình độ người học, lứa tuổi, dạy từ dễ đến khó; phải kết hợp học tập với vui chơi giải trí lành mạnh.\\
				Phương pháp hợp lý nhất là nêu ngương: mỗi thầy cô giáo là một tấm gương,...
			\end{itemize}
			\item Đội ngũ giáo viên: phải có đạo đức cách mạng, gỏi về chuyên môn và thuần thục về phương pháp. Mỗi giáo viên phải là một tấm gương sáng về đạo đức, về học tập.
		\end{itemize}
		\item Văn hóa văn nghệ
		Văn nghệ bao gồm văn học và nghệ thuật. Văn hóa văn ngh65 phải thực hiện được ba chức năng của văn hóa.\\
		Quan điểm của Hồ Chí Minh về văn hóa văn nghệ:
		\begin{itemize}
			\item Văn hóa văn nghệ là một mặt trận, nghệ sĩ là chiến sĩ, tác phẩm là vũ khí sắc bén trong đấu tranh cách mạng.\\
			Là một mặt trận, ở đó sẽ có tính chiến đấu và là đấu tranh rất gay go. Do vậy, để hoàn thành nhiệm vụ, "chiến sĩ nghệ thuật cần có lập trường vững, tư tưởng đúng,... đặt lợi ích của kháng chiến, của Tổ quốc, của nhân dân lên trên hết, trước hết".
			\item Văn nghệ phải gắn với thực tiễn đời sống của nhân dân.
			\item Phải có những tác phẩm văn nghệ xứng đáng với thời đại mới của đất nước của dân tộc.
		\end{itemize}
		\item Văn hóa đời sống\\
		Văn hoá đời sống là thực chất là đời sống mới: đạo đức mới, lối sống mới, nếp sống mới.
		\begin{itemize}
			\item Đạo đức mới: mỗi cá nhân cần thực hiện cần kiệm liêm chính. "Nêu cao và thực hành Cần, Kiệm, Liêm, Chính tức là nhen lửa cho đời sống mới".
			\item Lối sống mới hay phong cách sống, phong cách làm việc là lối sống có lý tưởng, cò đạo đức.
			\begin{itemize}
				\item Phong cách sống phải có lý tưởng, có đạo đức.
				\item Phong cách làm việc phải có tác phong quần chúng, tác phong tập thể - dân chủ, tác phong khoa học.
			\end{itemize}
			\item Nếp sống mới: hay phong cách cộng đồng cần phù hợp với truyền thống, với thuần phong mỹ tục.
		\end{itemize}
	\end{enumerate}
\end{document}

